\documentclass[11pt,a4paper]{ctexart}

\usepackage{subfigure}
\usepackage[graphicx]{realboxes}
\usepackage{listings}
\usepackage{xcolor}
\usepackage{amsmath}

% (1) choose a font that is available as T1
\usepackage{lmodern}
% (2) specify encoding
\usepackage[T1]{fontenc}
% (3) load symbol definitions
\usepackage{textcomp}

\usepackage{hyperref}

\usepackage{graphicx}

\usepackage{tabularx}
\usepackage{array}


\hypersetup{hidelinks,
	colorlinks=true,
	allcolors=black,
	pdfstartview=Fit,
	breaklinks=true}


\definecolor{mygrey}{rgb}{0.945,0.945,0.945}
\definecolor{myred}{rgb}{1, 0.49, 0.63}
\definecolor{myblue}{rgb}{0,0.36, 0.67}

\lstset{
	breaklines=true,
	basicstyle=\fontspec{Consolas},
	columns=fixed,       
	numbers=left,                                        % 在左侧显示行号
	numberstyle=\tiny\color{gray},                       % 设定行号格式
	frame=none,                                          % 不显示背景边框
	backgroundcolor=\color[RGB]{245,245,244},            % 设定背景颜色
	keywordstyle=\color[RGB]{40,40,255},                 % 设定关键字颜色
	numberstyle=\footnotesize\color{darkgray},           
	commentstyle=\it\color[RGB]{0,96,96},                % 设置代码注释的格式
	stringstyle=\rmfamily\slshape\color[RGB]{128,0,0},   % 设置字符串格式
	showstringspaces=false,                              % 不显示字符串中的空格
	language=c++,                                        % 设置语言
}
%\def\inline{\lstinline[basicstyle=\fontspec{微软雅黑},keywordstyle={}]}

%opening
\title{汇编}
\author{Liam}

\begin{document}
\maketitle

\section{基础知识}
\subsection{存储器}
存储器中存储了\textbf{指令}和\textbf{数据},同时存储器又被划分为若干个存储单元。
CPU对数据进行读写需要获取:地址信息,控制信息,数据信息。通过:\textbf{地址总线},\textbf{控制总线}和\textbf{数据总线}进行传输。

\subsubsection{地址总线}
一个CPU有N根地址线,则该CPU的地址总线的宽度为N,这样CPU可以寻址$2^N$个内存单元。注意:每个内存单元可以容纳8位的数据比特。

\subsection{数据总线}
数据总线宽度决定了数据传输速度。

\subsection{控制总线}
控制总线宽度决定了CPU对外部器件的控制能力。

\subsection{内存地址空间}
各种物理器件,在CPU操控的时候,都是将他们看作内存来对待---所有物理存储器被看作\textbf{一个由若干存储单元组成的逻辑存储器}。每个物理存储器在这个逻辑存储器中占有一个地址段,
即一段地址空间。CPU向对应的地址读写数据,相当于向对应的物理存储器中读写数据。

\section{寄存器}
CPU由原算起,控制器和寄存器等器件构成,二程序员通过改变寄存器中的内容来实现对CPU的控制。
\subsection{通用寄存器}
一个16位寄存器可以存储16位的数据,并且有些可以将16位寄存器拆成\textbf{高八位寄存器}和\textbf{低八位寄存器}使用。

\subsection{物理地址}
么一个内存单元在内存空间中都有唯一的地址,我们称为\textbf{物理地址}。

\subsection{16位结构的CPU}
\begin{itemize}
	\item 运算器一次处理16位数据
	\item 寄存器最大宽度为16位
	\item 寄存器和运算器之间通路为16位
\end{itemize}
\subsection{16位8086CPU给出20位寻址能力的办法}
采用公式(即段地址左移4位):
\begin{equation}
	\text{物理地址}=\text{段地址}\times 16 +\text{偏移地址}
\end{equation}
来合成20位物理地址。(注意本身内存是没有被划分成一段段的,只是人们为了管理而虚指的)
\subsection{段寄存器}
\subsubsection{CS和IP}
CS为段寄存器,IP为指令指针寄存器。在8086PC机中,任意时刻CS中内容为M,IP中内容为N,则CPU将会从$M\times 16+N$单元开始,
读取一条指令并执行。
~\\
步骤:
\begin{enumerate}
	\item \colorbox{mygrey}{\color{myblue}CS:IP} 指向的内存单元读取指令,指令进入指令缓冲区
	\item IP=IP+所读取的指令的长度并指向下一条指令
	\item 执行指令,回到步骤1
\end{enumerate}

\subsection{修改CS,IP指令}
通过指令 $jmp \text{段地址}:\text{偏移地址}$  来将修改CS和IP。
\par
通过指令 $jmp 某一合法寄存器$  ,来将修改CS和IP,所以任意代码段的执行,只能依靠CS:PC来确定执行。

\end{document}
